\documentclass[10pt]{report}

\usepackage{stan-talks}

\begin{document}
\sf%
\mbox{ }
\\[12pt]
\spc{\LARGE\bfseries \color{MidnightBlue}{A meta-analysis of
    SARS-CoV-2 prevalence}}
\\[4pt]
\spc{\large \color{Black}{using the Stan probabilistic
    programming language}}
\\[36pt]
\noindent
\spc{\large\bfseries \color{MidnightBlue}{Bob Carpenter}}
\\[2pt]
\spc{\normalsize Center for Computational Mathematics, Flatiron Institute}
\vfill
\noindent
\spc{\small May 2020} \hfill
\hfill
\includegraphics[width=0.4in]{img/new-logo.png}

\sld{What is prevalence?}
\begin{itemize}
\item A condition's \myemph{prevalence} is the proportion of the
  population that has it
\begin{subitemize}
\item e.g., if 32 of a population of 1000 has a condition, its
  prevalence is 3.2\%.
\end{subitemize}
\item We'd like to \myemph{estimate} prevalence of individuals
\begin{enumerate}
\item with SARS-Cov-2 \myemph{virus},
\item with COVID-19 \myemph{disease},
\item who have developed \myemph{antibodies} to SARS-Cov-2, and
\item who are \myemph{infectious}.
\end{enumerate}
\item Viral infection (1) is the focus of this talk
\end{itemize}

\sld{Why is estimation challenging?}
\begin{itemize}
\item Conditions form multiple \myemph{scales}
\begin{subitemize}
\item how much virus? which symptoms? how infectious? which antibodies?
\end{subitemize}
\item Measurements are \myemph{noisy}
\begin{subitemize}
\item \myemph{error}: inaccurate tests, varying accuracy across sites, human
  judgement, $\ldots$
\item \myemph{sampling}: extrapolate from sample to population
\end{subitemize}
\item Population \myemph{heterogeneity}
\begin{subitemize}
\item \myemph{demographics}: sex, age, existing medical conditions $\ldots$
\item \myemph{behavior}: social distancing, protective measures, food, travel, $\ldots$
\item \myemph{geo-political}: location, (local) government, climate, $\ldots$
\item \myemph{temporal}: prevalance evolves over time
\item \myemph{testing}: availability, assignment, self selection, $\ldots$
\end{subitemize}
\end{itemize}

\sld{Understanding sampling uncertainty}
\begin{itemize}
\item \myemph{Simulate}: false positive results; N = 100, 2\% false
  positive rate
\begin{subitemize}
\item \myemph{simulated false positives} (100 simulations): \footnotesize{1 2 1 2 0 2 4
    4 2 3 3 2 3 0 1 1 1 4 2 1 1 4 0 1 1 3 1 0 2 1 8 2 4 2 2 4 1 4
    0 1 0 0 3 1 5 1 3 3 4 0 3 5 0 3 1 3 2 3 1 0 1 4 2 2 1 0 2 1 1 1 2
    1 1 3 2 2 3 2 0 1 2 3 1 1 1 2 2 0 2 4 2 2 2 3 3 1 1 4 3 2}
\item \normalsize \myemph{min} 0 (0\%); \ \myemph{max} 8 (8\%); \
  \myemph{std dev} 1.4 (1.4\%)
\end{subitemize}
\item \myemph{Simulate}: positive status; N = 3000, 1.5\% prevalence
\begin{subitemize}
\item \myemph{simulated positives} (100 simulations): \footnotesize{39 51 42 43 52 52 37 47
    41 51 43 47 47 41 49 43 40 44 46 44 49 50 54 48 31 44 57 40 46 40
    51 49 48 46 51 40 47 47 42 42 42 40 55 34 40 48 35 39 45 48 42 42
    45 54 43 40 40 39 48 42 45 36 41 47 40 42 43 41 39 52 47 46
    43 38 46 31 49 27 39 42 43 46 37 38 36 45 36 47 41 35 49 43 51 45
    47 34 46 43 46 49}
\item \normalsize \myemph{min} 27 (0.9\%); \ \myemph{max} 57 (1.9\%); \
  \myemph{std dev} 5.5 (0.2\%)
\end{subitemize}
\end{itemize}

\sld{Sensitivity and specificity of diagnostic tests}

\begin{itemize}
\item Split accuracy based on status of individuals to account for
  test biases
\item \myemph{sensitivity} is accuracy with positive status
  \hfill  $\textrm{Pr}[\textrm{test} = 1 \mid \textrm{status} = 1]$
\begin{subitemize}
\item sensitive tests have low false negative rates
\end{subitemize}
\item \myemph{specificity} is accuracy on negative status
\hfill $\textrm{Pr}[\textrm{test} = 0 \mid \textrm{status} = 0]$
\begin{subitemize}
\item specific tests have low false negative rates
\end{subitemize}
\item Examples from breast cancer diagnosis
\begin{subitemize}
\item \myemph{mammogram, MRI}: high sensitivity, low specificity
\item \myemph{puncture biopsy}: low sensitivity, high specificity
\item this profile can't catch breast cancer reliably until it's \myemph{too late}
\end{subitemize}
\end{itemize}

\sld{Analyzing Serum PCR tests for SARS-CoV-2}
\begin{itemize}
\item \myemph{Sensitivity} tests (known positives)
\hfill \myemph{Specificity} tests (known negatives)
\begin{subitemize}
\item[]
{\small
\begin{tabular}[t]{rr|r}
positives & total & sensitivity \\ \hline
78 & 85 & 92\% \\
27 & 37 & 73\% \\
25 & 35 & 71\% \\
\end{tabular}
}
\hfill
{\footnotesize
\begin{tabular}[t]{rr|r}
negatives & total & specificity \\ \hline
368     & 371   & 99\% \\
30      & 30    & 100\% \\
70      & 70    & 100\% \\
1102    & 1102  & 100\% \\
300     & 300   & 100\% \\
311     & 311   & 100\% \\
500     & 500   & 100\% \\
198     & 200   & 99\% \\
99      & 99    & 100\% \\
29      & 31    & 94\% \\
146     & 150   & 97\% \\
105     & 108   & 97\% \\
50      & 52    & 96\% \\
\end{tabular}
}
\end{subitemize}
\vspace*{-84pt}
\item \myemph{Prevalence test} (unknown status)
\begin{subitemize}
\item[]
{\small
\begin{tabular}[t]{rr|r}
positives & total & prevalence \\ \hline
50 & 3300 & 1.5\%
\end{tabular}
}
\end{subitemize}
\vspace*{12pt}
\item \normalsize \myemph{Goal}: estimate of \myemph{SARS-Cov-2 prevalence}
\end{itemize}

\sld{Adjust for test sensitivity \& specificity}
\begin{itemize}
\item Proportion of positive tests in sample must be \myemph{adjusted}.
\begin{subitemize}
\item for test sensitivity and specifity
\end{subitemize}
\item Expected \myemph{proportion of positive tests} is
\begin{eqnarray*}
\textsf{Pr}[\textsf{test} = 1]
& = &
\begin{array}[t]{l}
\textsf{Pr}[\textsf{status} = 1]
  \times
  \textsf{Pr}[\textsf{test} = 1 \mid \textsf{status} = 1]
\\
\ { } +
\textsf{Pr}[\textsf{status} = 0]
  \times
  \textsf{Pr}[\textsf{test} = 1 \mid \textsf{status} = 0]
\end{array}
\\[6pt]
& = &
\textsf{prev} \times \textsf{sens}
+ (1 - \textsf{prev}) \times (1 - \textsf{sens}).
\end{eqnarray*}
\item \myemph{Solve for expected prevalence} given sensitivity, specificity, positive
  tests.
$$
\textsf{prev}
= \frac{\displaystyle \textsf{pos} + \textsf{spec} - 1}
       {\displaystyle \textsf{sens} + \textsf{spec} - 1}
$$
\end{itemize}

\sld{Uncertainty behind prevalence estimates}
\begin{itemize}
\item Previous slide assumes sensitivity and specificity are known.
\item Three forms of uncertainty lead to uncertainty in prevalence:
\begin{subitemize}
\item test \myemph{sensitivity and specificity are unknown} and
  estimated from data,
\item the result of a \myemph{test is uncertain} given the status
  of an individual, and
\item tests are applied to \myemph{only a sample} of a population.
\end{subitemize}
\item The job of statistics is to \myemph{adjust for bias} and
  \myemph{quantify uncertainty}
\begin{subitemize}
\item it's \myemph{not magic}---it's \myemph{assumption driven}
\end{subitemize}
\end{itemize}

\sld{Test sensitivty and specificity varies by site}
\begin{itemize}
\item sensitivity and specificity are intrinsically \myemph{anti-correlated}
\begin{subitemize}
\item adjusting thresholds trades one for the other
\end{subitemize}
\item sensitivity and specificity are \myemph{correlated by site}
\begin{subitemize}
\item good procedures increase both; bad procedures decrease both
\end{subitemize}
\item perform a \myemph{meta-analysis} with a \myemph{hierarchical
    model} to
\begin{subitemize}
\item estimate \myemph{mean sensitivity and specificity} of the test,
\item estimate \myemph{each site's} sensitivity and specificity,
\item let amount of variation among sites control how much to
  \myemph{pool data}, and
\item predict behavior in new test sites with no control cases.
\end{subitemize}
\end{itemize}

\sld{Stan Code (Data \& Parameters)}
{\small
\begin{stancode}
data {                             parameters {
  int<lower = 0> K_pos;              real<lower = 0, upper = 1> prev;
  int<lower = 0> N_pos[K_pos];       vector<lower = 0, upper = 1> sens[K_pos];
  int<lower = 0> n_pos[K_pos];       vector<lower = 0, upper = 1> spec[K_neg];
  int<lower = 0> K_neg;              real<lower = 0, upper = 1> mu_sens;
  int<lower = 0> N_neg[K_neg];       real<lower = 0> kappa_sens;
  int<lower = 0> n_neg[K_neg];       real<lower = 0, upper = 1> mu_spec;
                                     real<lower = 0> kappa_spec;
  int<lower = 0> N_unk;              vector<lower = 0, upper = 1> sens_unk
  int<lower = 0> n_unk;              vector<lower = 0, upper = 1> spec_unk;
}                                  }
\end{stancode}
}

\sld{Stan Code (Model)}
{\small
\begin{stancode}
model {
  // hyperprior
  prev ~ uniform(0, 1);
  mu_spec, mu_sens ~ beta(9, 1);
  kappa_sens, kappa_spec ~ exponential(0.5);

  // prior (hierarchical)
  sens, suns_unk ~ beta(mu_sens * kappa_sens, (1 - mu_sens) * kappa_sens);
  spec, spec_unk ~ beta(mu_spec * kappa_spec, (1 - mu_spec) * kappa_spec);

  // likelihood
  n_pos ~ binomial(N_pos, sens);
  n_neg ~ binomial(N_neg, spec);
  n_unk ~ binomial(N_unk, prev * sens_unk + (1 - prev) * spec_unk);
}
\end{stancode}
}

\sld{Running Stan Code}
\begin{itemize}
\item Can be run from R, Python, Julia, MATLAB, Mathematica, or shell
\item[] Output for justthe prevalence estimate
{\small
\begin{stancode}
       mean  se_mean     sd   2.5%    50%  97.5%  n_eff  Rhat
prev  0.013        0  0.003  0.007  0.012  0.019   7795     1
\end{stancode}
}
\item \myemph{95\% posterior interval} is (0.007, 0.019)
\item Result is highly dependent on breadth of \myemph{sensitivity hyperprior}
\begin{subitemize}
\item only 3 sensitivity tests available
\end{subitemize}
\item Result does not vary among a range of \myemph{weakly regularizing} hyperpriors
\begin{subitemize}
\item e.g, assumiming variation among sites is on the order of
  1--20\%, but not 50\%.
\end{subitemize}
\item Assuming no variation \myemph{underestimates uncertainty}
\end{itemize}

\sld{Adjusting for non-representative samples}
\begin{itemize}
\item Prevalence \myemph{varies in subpopulations}
\begin{subitemize}
\item exposure risk by demographics; geographically by population
  density/travel; differing metabolism by age, sex; political
  and social effects
\end{subitemize}
\item May not have a random sample
\begin{subitemize}
\item because of purposeful stratified design; or convenience opt-in sample
\end{subitemize}
\item Either way, we use \myemph{multilevel regression} and \myemph{post-stratifification} to adjust
\begin{subitemize}
\item[] Step 0. fit a multilevel regression to the data (for regularization/pooling)
\item[] Step 1. estimate prevalence in each demographic subgroup
\item[] Step 2. weight prevalence in subgroups by their size
\end{subitemize}
\item Simulations in paper; real results awaiting Stanford IRB approval
\end{itemize}



\sld{Further Reading}
\begin{itemize}
\item \myemph{Project home page}: {\small\tt https://bob-carpenter.github.io/diagnostic-testing}
\item \myemph{Stan home page}: {\small\tt https://mc-stan.org}
\item \myemph{Reports} (comments welcome!)
\begin{subitemize}
\item
Gelman, A.\ \& B.\ Carpenter.\ 2020.\
Bayesian analysis of tests with unknown specificity and
sensitivity. {\slshape DRAFT.}
\item
Carpenter, B.\ \& A.\ Gelman.\ 2020.\ Case study of
  seroprevalence meta-analysis. {\slshape DRAFT.}
\item
Carpenter, B., A.\ Gelman, M.~D.\ Hoffman, et al.\ (2017). Stan:
  A probabilistic programming language. {\slshape J.\ Stat.\ Soft.}
  76(1).
\item
Carpenter, B.  2016.  Stan case study: Hierarchical partial pooling for repeated
  binary trials.  {\small\tt https://mc-stan.org/users/documentation/case-studies}
\end{subitemize}
\end{itemize}

\sld{Stan Availability and Usage}
\begin{subitemize}
\item \myemph{Platforms:} \ Linux, Mac OS X, Windows
\vspace*{-4pt}
\item \myemph{Interfaces:} \ R, Python, Julia, MATLAB, Mathematica
\vspace*{-4pt}
\item \myemph{Developers (academia \& industry):} 40+ \ {\small (15+ FTEs)}
\vspace*{-4pt}
\item \myemph{Users:}\ tens or hundreds of thousands
\vspace*{-4pt}
\item \myemph{Companies using:} \ hundreds or thousands
\vspace*{-4pt}
\item \myemph{Downloads:}\ millions
\vspace*{-4pt}
\item \myemph{User's Group:} \ 3000+ registered; 6000+ non-bot views/day
\vspace*{-4pt}
\item \myemph{Books using:} \ 10+
\vspace*{-4pt}
\item \myemph{Courses using:} \ 100+
\vspace*{-4pt}
\item \myemph{Case studies about:} 100+
\vspace*{-4pt}
\item \myemph{Articles using:} \ 5000+
\vspace*{-4pt}
\item \myemph{Conferences:} 4 (800+ attendance);  \myemph{StanCon 2020} will be online
\end{subitemize}

\sld{Some published applications of Stan}
%
\begin{subitemize}
\item \myemph{Physical sciences}: {\footnotesize
astrophysics, statistical mechanics, particle physics, organic
chemistry, physical ehmistry, geology, hydrology,
oceanography, climatology, biogeochemistry, materials science, $\ldots$
}
\vspace*{-3pt}
\item \myemph{Biological sciences}: {\footnotesize
molecular biology, clinical drug trials, entomology, pharmacology,
toxicology, opthalmology, neurology, genomics, agriculture, botany, fisheries,
epidemiology, population ecology, neurology, psychiatry, $\ldots$
}
\vspace*{-3pt}
\item \myemph{Social sciences}: {\footnotesize
 econometrics (macro and micro), population dynamics, cognitive
 science, psycholinguistics, social networks, political science,
 survey sampling, anthropology, sociology, social work, $\ldots$
}
\vspace*{-3pt}
\item \myemph{Other}: {\footnotesize education, public health, A/B testing,
government, finance, machine learning, transportation logistics,
electrical engineering, mechanical engineering, civil engineering and transportation,
actuarial science, sports analytics, advertising attribution, marketing, $\ldots$}
\end{subitemize}

\sld{Industries using Stan}
\vspace*{3pt}
\begin{subitemize}
\item \myemph{Marketing attribution}: Google, Domino's Pizza, Legendary Ent.
\item \myemph{Demand forecasting}: Facebook, Salesforce
\item \myemph{Financial modeling}: Two Sigma, Point72
\item \myemph{Pharmacology \& CTs}: Novartis, Pfizer, Astra Zeneca
\item \myemph{(E-)sports analytics}: Tampa Bay Rays, NBA, Sony Playstation
\item \myemph{Survey sampling}: YouGov, Catalist
\item \myemph{Agronomy}: Climate Corp., CiBO Analytics
\item \myemph{Real estate pricing models}: Reaktor
\item \myemph{Industrial process control}: Fero Labs
\end{subitemize}


\sld{Why is Stan so Popular?}
\vspace*{3pt}
\begin{subitemize}
\item \myemph{Community}: large, friendly, helpful, and sharing
\item \myemph{Documentation}:  novice to expert; breadth of fields
\item \myemph{Robustness}:  industrial-strength code; user diagnostics
\item \myemph{Flexibility}:  highly expressive language;  large math lib
\item \myemph{Portability}: popular OS, language, and cloud support
\item \myemph{Extensibility}: developer friendly; derived packages
\item \myemph{Speed}:  $2-\infty$ orders of magnitude faster
\item \myemph{Scalability}:  2+ orders of magnitude more scalable
\item \myemph{Openness}: permissive code and doc licensing
\end{subitemize}

\end{document}


\mypart{}{Motivation}
\pagestyle{plain}

\sld{Scientists}
\pagestyle{plain}
\begin{itemize}
\item want to \myemph{understand} the world,
\item bring mechanistic \myemph{theories},
\item bring data in the form of \myemph{measurements}, and
\item need to make \myemph{predictions} and \myemph{evaluate theories}.
\end{itemize}

\sld{Policymakers}
\begin{itemize}
\item want to \myemph{control} the social world,
\item bring hypotheses about \myemph{interventions},
\item bring data in the form of \myemph{measurements}, and
\item need to \myemph{make decisions} and \myemph{evaluate impact}.
\end{itemize}

\sld{Uncertainty}
\begin{itemize}
\item \myemph{permeates science} and decision making
\item in the form of
\begin{subitemize}
\item \myemph{sampling} uncertainty,
\item \myemph{measurement} uncertainty, and
\item \myemph{modeling} uncertainty.
\end{subitemize}
\vfill
\item The \myemph{alternative to good statistics} is
\begin{subitemize}
\item not no statistics,
\item but \myemph{bad statistics}. \hfill (Bill James)
\end{subitemize}
\end{itemize}

\sld{Probability \& Statistics}
\begin{itemize}
\item Probability theory uses math to \myemph{quantify uncertainty}.
\item \myemph{Bayesian statistics} applies probability theory to
\begin{subitemize}
\item data {analysis},
\item {prediction} \& forecasting,
\item model {evaluation}, and
\item {decision} theory.
\end{subitemize}
\end{itemize}

\sld{This talk}
\begin{enumerate}
\item How to combine data, theory, and probability to
\begin{subitemize}
\item make \myemph{calibrated predictions},
\item \myemph{optimal decisions} under uncertainty, and
\item along the way, solve the \myemph{replication crisis}.
\end{subitemize}
\item \myemph{Stan}, an open probabilistic programming framework
supporting {Bayesian workflow}
\item Highlights of my \myemph{current projects}
\end{enumerate}

\mypart{}{Stan}

\sld{What is Stan?}
%
\begin{itemize}
\item a domain-specific \myemph{probabilistic programming language}
\item Stan \myemph{program} defines a \myemph{differentiable} probability model
  \begin{subitemize}
  \item declares data and (constrained) parameter variables
  \item defines log posterior (or penalized likelihood)
  \item defines predictive quantities
  \end{subitemize}
\item Stan \myemph{inference} fits model \& makes predictions
  \begin{subitemize}
  \item MCMC for full Bayesian inference
  \item variational and Laplace for approximate Bayes
  \end{subitemize}
\item \footnotesize Carpenter, B., Gelman, A., Hoffman, M. D., Lee, D., Goodrich, B., Betancourt, M., Marcus Brubaker, Jiqiang Guo, Peter Li, Riddell, A. (2017). Stan: A probabilistic programming language. {\slshape J.\ Stat.\ Soft.} 76(1).
\end{itemize}



\end{document}
